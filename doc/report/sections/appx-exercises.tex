% !TEX root = ../main.tex

% Exercises section

\section{Exercises}

\begin{enumerate}
	\item 
	Consider a probability measure $\mu$ on a group $G$ that acts on the real numbers $\mathbb{R}$ equipped with the Borel $\sigma$-algebra $\borel(\bbR)$.
	For $x \in \mathbb{R}$, define the $\sigma(\mathcal{G})$-measurable function $h_x: G \to \bbR$ defined by $h_x(g) := f(\bbT_g(x))$ for some $\borel(\bbR)$-measurable function $f$.
	Compute $\int_{G} h_x(g) d\mu(g)$ under the following conditions.
	\begin{enumerate}[label=(\alph*)]
		\item
		$f$ is $G$-invariant; that is, $f(\bbT_g(x)) = f(x)$ for all $g$.
		
		\item $f$ is $G$-equivariant, $G = \bbR \setminus \{0\}$, and $\bbT_g(x) = gx$.
	\end{enumerate}
	
	\item
	Suppose $Y \condind X$.
	Provide $f$ so that $Y \equdist f(\eta,X)$ conditioned on $X$ for some $\eta \sim \UnifDist[0,1]$.
	
	\item Consider a group $(G, \cdot)$ and a set $X$ upon which it acts through its group action $\bbT$.
	Prove that the stabilizer of $x$ defined by
	$$
		G_x := \{ g \in G : \bbT(g,x) = x \}
	$$
	forms a group with the binary operator $\cdot$ borrowed from $G$.
\end{enumerate}

\pagebreak

\section{Solution to exercises}

\begin{enumerate}
\item
\begin{enumerate}[label=(\alph*)]
	\item
	We have
	$$
	\begin{aligned}
		\int_{G} h_x(g) d \mu(g) & = \int_G f(\bbT_g(x)) d \mu(g) & \textrm{by definition of $h_x$} \\
		                         & = \int_G f(x) d \mu(g) & \textrm{by $G$-invariance of $f$} \\
		                         & = f(x) \int_G d \mu(g) & \textrm{$f(x)$ is a constant with respect to the integral} \\
		                         & = f(x) \mu(G) \\
		                         & = f(x)
	\end{aligned}
	$$
	
	\item
	Define $Y$ to be a random variable whose distribution is $\mu$.
	We have
	$$
	\begin{aligned}
		\int_G h_x(g) d \mu(g) & = \int_G f(\bbT_g(x)) d \mu(g) \\
		                       & = \int_G \bbT_g(f(x)) d \mu(g) & \textrm{the action of $G$ on the range of $f$ is simply the original action $\bbT_g$} \\ 
		                       & = \int_G g f(x) d \mu(g) \\
		                       & = f(x) \int_G g\ d \mu(g) \\
		                       & = f(x) \bbE[Y]
	\end{aligned}
	$$
\end{enumerate}

\item 
Because $X \indep Y$ and we assume that $X \indep \eta$, the quantile function $Q$ of $Y$ will do, so that
$
	Y \equdist Q(\eta).
$

\item We will show that $(G_x, \cdot)$ satisfies the properties of a group described in definition \ref{def:group}.
\begin{enumerate}
	\item
	Consider $u,v \in G_x$.
	Notice that
	$$
		\bbT(u \cdot v, x) = \bbT(u, \bbT(v,x)) = \bbT(u, x) = x.
	$$
	Therefore, $u \cdot v \in G_x$.
	
	\item
	By definition of group action, $\bbT(e,x) = x$, where $e$ is the identity.
	Therefore, $e \in G_x$.
	
	\item
	Suppose for contradiction that $g \in G_x$ but $g^{-1} \not \in G_x$.
	Then we have
	$$
	\begin{aligned} 
		\bbT(g^{-1},x) \neq x & \implies \bbT(g, \bbT(g^{-1}, x)) \neq \bbT(g, x) \\
		& \implies \bbT(g\cdot g^{-1}, x) \neq x \\
		& \implies \bbT(e, x) \neq x \\
		& \implies x \neq x
	\end{aligned}
	$$
	a contradiction.
\end{enumerate}
\end{enumerate}

\section{Theorem \ref{thm:convolution} in the language of Functional Analysis}
The following section references the textbook \cite{BuhlerTheo2018Fa/T}.
Theorem \ref{thm:convolution} states that if a random variable $Y$ is almost surely an $G$-equivariant linear function of $X$, then $Y$ is almost surely a convolution of $X$.
Just as how we can view $\bbR^n$ as representing some geometric \textit{vector space} with notions of length and orthogonality, we can think of vector \textit{spaces} of functions in the same way, by defining an \textit{inner product} on functions.

The most common inner product that you are probably aware of is the dot product on finite dimensional vector spaces denoted by $u \cdot v$.
We can extend this notion of inner product to more abstract spaces, even spaces of infinite dimension!
We say that a function $\langle \cdot, \cdot \rangle : V \times V \to \bbR$ is an inner product on a vector space $V$ if it satisfies the following properties for all vectors $u,v,w \in V$ and scalars $a \in \bbR$:
\begin{enumerate}[label=(\alph*)]
	\item symmetry: $\langle x, y \rangle = \langle y, x \rangle$,
	\item linearity in the first argument:
	$$
		\langle x + y, z \rangle = \langle x, z \rangle + \langle y,z \rangle,
	$$
	$$
		\langle ax, y \rangle = a \langle x,y \rangle
	$$
	and
	\item 
	positive-semidefiniteness:
	$$
		\langle x, x \rangle \geq 0.
	$$
	with $\langle x, x \rangle = 0$ if and only if $x = \textbf{0}$.
\end{enumerate}
One popular way to define an inner product on a space of \textit{functions} is through integration: given functions $f$ and $g$, the inner product of these two functions is given by
$$
	\langle f,g \rangle = \int_{\bbR} f(x) g(x)\ dx.
$$
Spaces of functions equipped with an inner product $\langle \cdot, \cdot \rangle$ are the central object of study in \textit{functional analysis}.

One important result in functional analysis is the \textit{Reisz representation theorem}, which states that (slightly paraphrased)
\begin{theorem}
Let $V$ be a vector space equipped with an inner product $\langle \cdot, \cdot \rangle$, and let $V^*$ be the space of all linear functions on $V$.
Then every linear function $f \in V^*$ can be written in the form
$$
	f(x) = \langle a, x \rangle
$$ 
for some vector $a \in V$.
\end{theorem}

With this in mind, we can rewrite theorem \ref{thm:convolution} in the language of functional analysis:

\begin{theorem}
	Consider random vectors $(X,Y)$ with index sets $\mathcal{I}_X$ and $\mathcal{I}_Y$ and a compact group $G$ which acts on $\mathcal{I}_X$ and $\mathcal{I}_Y$.
	Moreover, suppose that $P_{X,Y}$ is $G$-equivariant.
	Lastly, suppose that the action of $G$ on both $\mathcal{I}_X$ and $\mathcal{I}_Y$ is transitive.
	Then if, conditioned on $X$, $Y \equas f(\eta, X)$ is the \textbf{inner product between $X$ and some vector $a(\eta)$ in a vector space $V$ almost surely} (i.e. for some almost sure subset of the range of $\eta$), then $Y \equas X * \chi(\eta)$ in the sense of definition \ref{def:convolution}.
	More generally, $(X,Y) \equas (X,X * \chi(\eta))$.
\end{theorem}