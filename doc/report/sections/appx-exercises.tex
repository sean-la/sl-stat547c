% !TEX root = ../main.tex

% Exercises section

\section{Exercises}

\begin{enumerate}
	\item 
	Consider a probability measure $\mu$ on a group $G$ that acts on the real numbers $\mathbb{R}$ equipped with the Borel $\sigma$-algebra $\borel(\bbR)$.
	For $x \in \mathbb{R}$, define the $\sigma(\mathcal{G})$-measurable function $h_x: G \to \bbR$ defined by $h_x(g) := f(\bbT_g(x))$ for some $\borel(\bbR)$-measurable function $f$.
	Compute $\int_{G} h_x(g) d\mu(g)$ under the following conditions.
	\begin{enumerate}[label=(\alph*)]
		\item
		$f$ is $G$-invariant; that is, $f(\bbT_g(x)) = f(x)$ for all $g$.
		
		\item $f$ is $G$-equivariant, $G = \bbR$, and $\bbT_g(x) = gx$.
	\end{enumerate}
	
	\item
	Suppose $Y \condind X$.
	Provide $f$ so that $Y \equdist f(\eta,X)$ conditioned on $X$ for some uniformly distributed $\eta$.
	
	\item Consider a group $(G, \cdot)$ and a set $X$ upon which it acts through its group action $\bbT$.
	Prove that the stabilizer of $x$ defined by
	$$
		G_x := \{ g \in G : \bbT(g,x) = x \}
	$$
	forms a group with the binary operator $\cdot$ borrowed from $G$.
\end{enumerate}

\pagebreak

\section{Solution to exercises}

\begin{enumerate}
\item
\begin{enumerate}[label=(\alph*)]
	\item
	We have
	$$
	\begin{aligned}
		\int_{G} h_x(g) d \mu(g) & = \int_G f(\bbT_g(x)) d \mu(g) & \textrm{by definition of $h_x$} \\
		                         & = \int_G f(x) d \mu(g) & \textrm{by $G$-invariance of $f$} \\
		                         & = f(x) \int_G d \mu(g) & \textrm{$f(x)$ is a constant with respect to the integral} \\
		                         & = f(x) \mu(G) \\
		                         & = f(x)
	\end{aligned}
	$$
	
	\item
	Define $Y$ to be a random variable whose distribution is $\mu$.
	We have
	$$
	\begin{aligned}
		\int_G h_x(g) d \mu(g) & = \int_G f(\bbT_g(x)) d \mu(g) \\
		                       & = \int_G \bbT_g(f(x)) d \mu(g) & \textrm{the action of $G$ on the range of $f$ is simply the original action $\bbT_g$} \\ 
		                       & = \int_G g f(x) d \mu(g) \\
		                       & = f(x) \int_G g\ d \mu(g) \\
		                       & = f(x) \bbE[Y]
	\end{aligned}
	$$
\end{enumerate}

\item 
Because $X \indep Y$ and we assume that $X \indep \eta$, the quantile function $Q$ of $Y$ will do, so that
$
	Y = Q(\eta).
$

\item We will show that $(G_x, \cdot)$ satisfies the properties of a group described in definition \ref{def:group}.
\begin{enumerate}
	\item
	Consider $u,v \in G_x$.
	Notice that
	$$
		\bbT(u \cdot v, x) = \bbT(u, \bbT(v,x)) = \bbT(u, x) = x.
	$$
	Therefore, $u \cdot v \in G_x$.
	
	\item
	By definition of group action, $\bbT(e,x) = x$, where $e$ is the identity.
	Therefore, $e \in G_x$.
	
	\item
	Suppose for contradiction that $g \in G_x$ but $g^{-1} \not \in G_x$.
	Then we have
	$$
	\begin{aligned} 
		\bbT(g^{-1},x) \neq x & \implies \bbT(g, \bbT(g^{-1}, x)) \neq \bbT(g, x) \\
		& \implies \bbT(g\cdot g^{-1}, x) \neq x \\
		& \implies \bbT(e, x) \neq x \\
		& \implies x \neq x
	\end{aligned}
	$$
	a contradiction.
\end{enumerate}
\end{enumerate}