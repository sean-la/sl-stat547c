% !TEX root = ../main.tex

% open questions section

\section{Open questions and research directions}
This research leads to several open questions:
\begin{enumerate}
\item  
We have shown that linearity of $f(\eta, \cdot)$ in the second argument along with equivariance of $f$ are sufficient for $Y = f(\eta, X) = X * \chi(\eta)$ when the group $G$ is compact.
Can we expand these results for when the group $G$ is not compact?

A possible approach to answer this question is by restricting the possible forms of the filter $\chi$.

\item We have shown that the filter function is a function of the noise-outsourced variable $\eta$ so that $\chi = \chi(\eta)$.
From an applications point of view, does adding noise to filters in a convolutional neural network increase its performance?
For example, does adding noise to filters act as a form of regularization, similar to how drop-out layers in neural networks serve that purpose?

Here are some possible ways to incorporate noise into the filter $\chi$:
\begin{itemize}
	\item The value of $\eta \in [0,1]$ serves as a scaling factor of the value of $\chi$.
	\item If the value of $\eta$ is below a certain threshold, then some components of $\chi$ is set to $0$, so that certain neurons in the previous layer are not included in the current layer.
	\item The value of $\eta$ serves as the probability that certain neurons in the previous layer are dropped, in similar vein to dropout layers.
\end{itemize}
\end{enumerate}